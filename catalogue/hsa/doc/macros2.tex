\usepackage{xspace}
%same notation policy for each instance of figure, thm, etc
\newcommand{\myfig}{Fig.~}
\newcommand{\mytab}{Tab.~}
\newcommand{\mysec}{Sec.~}
\newcommand{\mychap}{Chap.~}
\newcommand{\myapp}{App.~}
\newcommand{\etal}{et~al.\xspace}
\newcommand{\wrt}{w.r.t.\xspace}
\newcommand{\eg}{e.g.\xspace}
\newcommand{\ie}{i.e.\xspace}
\newcommand{\etc}{etc.\xspace}
\newcommand{\cf}{see\xspace}

%les maths comme au lycee
\let\eset\emptyset
\newcommand{\rplus}[1]{\mathop{{#1}^{+}}}
\newcommand{\rstar}[1]{\mathop{{#1}^{*}}}
\newcommand{\transc}[1]{\mathop{{#1}^{+}}}
\newcommand{\maybe}[1]{\mathop{{#1}^{?}}}

%def
  \newcommand{\dfnshort}[2]{{#1}\triangleq{#2}}
  \newcommand{\dfn}[2]{{#1}~\triangleq~{#2}}

%sets
\newcommand{\setcal}[1]{\mathcal{#1} }
\newcommand{\setbf}[1]{\mathbb{#1} }
\newcommand{\setfrak}[1]{\mathfrak{#1} }


  %writes, reads, barriers
  \let\setset\setbf
  \newcommand{\evts}{\setset{E}}

  \newcommand{\lwf}{\textsf{fence}}
  \newcommand{\allf}{\textsf{fence.all}}
  \newcommand{\mfence}{\ensuremath{\mathsf{mfence}}}
  \newcommand{\lwfence}{\ensuremath{\mathsf{fence}}}
  \newcommand{\ff}{\textsf{fence.sc}}
  \newcommand{\ffence}{\ensuremath{\mathsf{f{\kern-0.1em}fence}}}
  \newcommand{\fences}{\ensuremath{\mathsf{fences}}}
  \newcommand{\relw}{\ensuremath{\mathsf{rel}}}
  \newcommand{\acqr}{\ensuremath{\mathsf{acq}}}
  \newcommand{\cumul}{\ensuremath{\mathsf{cumul}}}
  \newcommand{\Acumul}{\ensuremath{\mathsf{A{\kern-0.3em}-{\kern-0.3em}cumul}}}

  %procs
  \newcommand{\pr}{\operatorname{proc}}

  %locs
  \newcommand{\lo}{\ell}
  \newcommand{\loc}{\operatorname{addr}}

  %vals 
  \newcommand{\val}{\operatorname{val}}

%relations
\newcommand{\stacklabel}[1]
%{\stackrel{\smash{\scriptscriptstyle\#1}}}
{\stackrel{\smash{\scriptstyle\textnormal{#1}}}}
  %any

  %po
  \newcommand{\iico}{\ensuremath{\mathsf{ii}}}
  \newcommand{\ctrlcfence}{\ensuremath{\mathsf{ctrlcfence}}}
  \newcommand{\ctrl}{\ensuremath{\mathsf{ctrl}}}
  \newcommand{\data}{\ensuremath{\mathsf{data}}}
  \newcommand{\addr}{\ensuremath{\mathsf{addr}}}
  \newcommand{\ddreg}{\ensuremath{\mathsf{dd}}}
  \newcommand{\rfreg}{{\mathsf{rfreg}}}
  \newcommand{\isb}{\ensuremath{\mathsf{isb}}}
  \newcommand{\isync}{\ensuremath{\mathsf{isync}}}
  \newcommand{\ppo}{\ensuremath{\mathsf{ppo}}}

\newcommand{\wide}{wide}
\newcommand{\wider}{wider}
\newcommand{\widest}{widest}
\newcommand{\narrow}{narrow}
\renewcommand{\narrower}{narrower}
\newcommand{\narrowest}{narrowest}
\newcommand{\narrowerop}{\operatorname{narrower}}

%%%Litmus
%%basic format
\let\prog\textsf
\let\as\texttt
\let\ltest\textbf

%% Conventional processor
\newcommand{\proc}[1]{\ensuremath{P_{#1}}}
%% Format graphs in columns
\makeatletter
\@ifundefined{useepstrue}{\newif\ifuseeps\useepsfalse}{}
\makeatother
\newlength{\fmtlength}
\ifuseeps
\newcommand{\fmtgraphcol}[3]
{\setlength{\fmtlength}{\linewidth}%
\addtolength{\fmtlength}{-#2}%
\framebox{\resizebox{\fmtlength}{#3}{\includegraphics{#1.eps}}}}
\else
\newcommand{\fmtgraphcol}[3]
{\setlength{\fmtlength}{\linewidth}%
\addtolength{\fmtlength}{-#2}%
\framebox{\resizebox{\fmtlength}{#3}{\input{#1.pstex_t}}}}
\fi
%%Format graph explicit width
\ifuseeps
\newcommand{\fmtgraph}[3]
{\resizebox{#2}{#3}{\includegraphics{#1.eps}}}
\newcommand{\tfmtgraph}[1]{\framebox{\includegraphics{#1.eps}}}
\else
\newcommand{\fmtgraph}[3]
{\resizebox{#2}{#3}{\input{#1.pstex_t}}}
\newcommand{\tfmtgraph}[1]{\framebox{\input{#1.pstex_t}}}
\fi
%%%%Test for figures as in the tutorial
% fix up fig2dev fonts to make actions in \sf
\iftrue
\makeatletter
% for the fig2dev version on P laptop
\gdef\SetFigFontNFSS#1#2#3#4#5{%
  \reset@font\fontsize{#1}{#2pt}%
%  \fontfamily{#3}\fontseries{#4}\fontshape{#5}%
  \fontfamily{\sfdefault}\fontseries{#4}\fontshape{#5}%
  \selectfont}%
% for the fig2dev version on P desktop
\gdef\SetFigFont#1#2#3#4#5{%
  \reset@font\fontsize{#1}{#2pt}%
%  \fontfamily{#3}\fontseries{#4}\fontshape{#5}%
  \fontfamily{\sfdefault}\fontseries{#4}\fontshape{#5}%
  \selectfont}
\makeatother
\newcommand{\asm}[1]{\texttt{#1}}
\newcommand{\mylegend}[2]{#1}
\ifbw\newcommand{\bw}{-bw}\else\newcommand{\bw}{}\fi
\ifuseeps
\newcommand{\locfmt}[1]{\includegraphics{#1.eps}}
\newcommand{\newfmt}[1]{\includegraphics{img/#1.eps}}
\newcommand{\insfmt}[1]{\scalebox{0.28}{\includegraphics{ins/#1.eps}}}
\else
\newcommand{\locfmt}[1]{\input{#1.pstex_t}}
\newcommand{\newfmt}[1]{\input{img/#1\bw.pstex_t}}
\newcommand{\insfmt}[1]{\scalebox{0.28}{\input{ins/#1\bw.pstex_t}}}
\fi
\fi

\newcommand{\NEW}[1]{\textcolor{blue}{#1}}

\newcolumntype{Y}{@{}r@{\,}X}
%%Decorate instruction with a label
\newcommand{\instab}[2]{\ \(#2\)  & \as{#1}}
%%Pseudo-code load & store
\newcommand{\pset}[2]{\(\as{#1} \leftarrow \as{#2}\)}
\newcommand{\pstore}[2]{\pset{#2}{#1}}
\newcommand{\pload}[2]{\pset{#1}{#2}}
%%trick to add vertical space somewhere
\newcommand{\haut}{\rule{0ex}{2ex}}
\newcommand{\bas}{\rule[-1ex]{0.5ex}{0ex}}
  \newcommand{\acyclic}{\operatorname{acyclic}}
  \newcommand{\reflexive}{\operatorname{reflexive}}
  \newcommand{\irrefl}{\operatorname{irreflexive}}

