\section{Examples}

Here we show additional examples. 
%The first $5$ ones are classical ones that we
%feel are worth mentioning. 
They are slightly more involved scenarios coming from discussions with the HSA
committee.

%\subsection{SC per location}

%\begin{figure}[htp]
%\caption{\label{hsa13} Test~\ltst{coWW} is forbidden.}
%\begin{center}\moveback\fmt{coww}
%\end{center}
%\end{figure}

%\subsection{Tony1}

%Tony proposed the following scenario:
%\begin{quotation}
%o   Assuming work-items release to agent scope when complete:
%
%�  dispatch work items could do an atomic\_st\_screl\_agent to hidden variable
%provided by packet processor
%
%�         probably do a rmw to a single counter
%
%�  packet processor does atomic\_ld\_acq\_agent on that variable so gets all data
%written by dispatch
%
%�         probably spin until counter to wait for whole dispatch to complete
%
%�  packet processor would do memfence\_screl\_system if indicated by packet release fence
%
%�  packet processor then does an atomic\_st\_rlx\_system to completion signal
%
%�  host (or other) thread does atomic\_ld\_rlx\_system on completion signal
%
%�  host (or other) thread can do a memfence\_scacq\_system if want to access generated data of dispatch
%
%o   Does this model result in correct semantics given current memory model?
%\end{quotation}

%\begin{figure}[!h]
%\setlength{\columnseprule}{.5pt}
%\begin{multicols}{2}
%\small
%\verbatiminput{img/t1-alt.litmus}
%\end{multicols}
%\caption{A tentative packet processor fence scenario\label{ppf}}
%\end{figure}
%{\color{blue} We implement it as shown in Figure~\ref{ppf}.  First, does the
%test match the intent? 
%This test implements the following scenario: \myth{0}
%sends the value $1$ to \myth{1} via $x$ and to \myth{2} via $y$. Then both {\tt
%P1} and {\tt P2} make calculations over $x$ and $y$ respectively, then
%increment the counter $c$ atomically. When the counter has reached $2$,
%\myth{0} accesses $x$ and $y$, and writes their updated values in the variables
%{\tt d1} and {\tt d2}. After doing a release fence at system scope, \myth{0}
%then sends a signal to \myth{3} via variable $s$; \myth{3} can then read {\tt
%d1} and {\tt d2} after having done an acquire fence at system scope.
%
%Our tool produces the same outcomes on the SC and HSA models, which means that
%this test should have enough synchronisation to behave as on SC.  The figures
%\ref{t100} and~\ref{t101} depict the two SC executions where \myth{3} reads the
%value~$1$ from the flag variable~\texttt{s}.  The two executions only differ by
%the order in which \myth{1} and \myth{2} increment the counter \texttt{c}.

%\begin{figure}
%\caption{\label{t100}First execution of test~\ltst{Tony1}}
%\begin{center}
%%\moveback\fmt{t1-00}
%\end{center}
%\end{figure}

%\begin{figure}
%\caption{\label{t101}Second execution of test~\ltst{Tony1}}
%\begin{center}
%%\moveback\fmt{t1-01}
%\end{center}
%\end{figure}
%\pagebreak

\subsection{Tony2}

Here we examine another test:
%proposed by Tony, from a question raised by Hakan:

{\footnotesize
\verbatiminput{img/t2.litmus}
}

The test condition \verb+(2:r2=0 \/ 2:r3=0)+ groups three possible test
outcomes: one of the register being zero and not the other, and both register
being zero.
%We first depict a candidate execution where both
%register hold a final value of zero.
%\begin{figure}
%\caption{\label{t2fig}One execution of test~\ltst{Tony2}}
%\begin{center}\moveback\fmt{t2}\end{center}
%\end{figure}
The test performs two release-acquire synchronisations and thus
resembles the test~\ltst{isa2} (Figure~\ref{isa2coh}),
with the following differences:
\begin{itemize}
\item Two variables (\texttt{x} and~\texttt{y}) are synchronised
where~\ltst{isa2} synchronised only one variable.
\item There is a fence on~\myth{0}, but it does not seem to contribute to
synchronisation, because deleting the fence does not change test behaviour on
our model.
\item The scope tree is different.  More precisely the test~\ltst{Tony2} has
two work-groups scope instances $\{\myth{0}, \myth{2}\}$ and $\{\myth{1}\}$.
That is, \myth{0} and~\myth{2} synchronise through a unit (\myth{1}) that is
not a member of their work-group. Nevertheless, all units belong to the same
agent, and all synchronising actions (events $c$, $d$, $e$ and~$f$) bear the
\texttt{agent} scope tag.
\end{itemize}
As a result, there are happens-before edges from the writes $a$ (to~\texttt{x})
and~$b$ (to \texttt{y}) by \myth{0} to the reads of the same locations
by~\myth{2} (namely $g$ and~$h$). This suffices to forbid the specified
executions as there is a contradiction with~\coh{}.

For analysing how the test is forbidden in detail we relax some checks,
\emph{i.e.} we run the model without performing some checks.  First, by
relaxing the consistency of $\hhb$ and $\coh$ (i.e. calling \prog{herd} with
option \texttt{-skipchecks HhbCohCons}), we get one extra outcome: 
\verb+2:r2=1 /\ 2:r3=0+. The corresponding execution is shown in Figure~\ref{t2hhb}.

\begin{figure}[H]
\caption{\label{t2hhb}Relaxing the consistency of \hhb{} and \coh{}}
\begin{center}\moveback
\fmt{t2+hhb}
\end{center}
\end{figure}
In other words, the location~\texttt{x}, which is accessed by ordinary load and
store instructions is synchronised solely by the means of the consistency of
\hhb{} and \coh{}.  The figure clearly shows the inconsistency:
$a \hhb h \coh a$.
It is important to note that relaxing the consistency of
$\hhb$ and~$\coh$ does not change $\hhb$,
although \hhb{} is defined in terms of a sub-relation~of~$\coh$.

However, event~$g$ still cannot read zero, or, equivalently,
register \verb+2:r2+ cannot contain zero at the end of execution.
This means that the behaviour is
still rejected by other checks of the model.  As a matter of fact all accesses
to location~\texttt{y} are atomic and even synchronising. 

We thus examine sequentially consistent orders. We first assume that, for a
given scope instance~$S$ the SC order $\SCS$ orders all synchronising events
in~$S$, regardless of their scope annotations (i.e. we formalise {\tt
sync-instances} using {\tt tag2scope}, see discussion in Section
\ref{sc-orders}).  It is then enough to relax the consistency of SC orders and
\coh{} to observe a value of zero for the read~$g$.
\begin{figure}[htp]
\caption{\label{t2agent}Relaxing further the consistency of SC orders and of~\coh{}}
\begin{center}\moveback
\fmt{t2+agent}
\end{center}
\end{figure}
The contradiction is visible on Figure~\ref{t2agent}: $b \SC{agent} g \coh b$.
Note that $b \SC{agent} g$ stems from the consistency of SC orders and~\hhb.

We now consider an alternative definition of the domain of
SC orders:  $\SCLVL{tag}$ now operates
on events that specify the scope~\text{tag} (or higher),
\emph{i.e.} we consider event scope annotations.
Such a restricted ``scope instance''
is depicted in figure~\ref{t2scope} as the equivalence relation
$\same{agent}$.
\begin{figure}[htp]
\caption{\label{t2scope}Restricted scope instance of level agent}
\begin{center}\moveback
\fmt{t2+scope}
\end{center}
\end{figure}
Note that neither~$b$ (write to~\texttt{y} by~\myth{0}) nor~$g$ (read
from~\texttt{y} by~\myth{2}) are part of the domain of~\SC{agent}. Thus
preventing $g$ from reading zero (\emph{i.e.} $g \coh b$) cannot come from a
contradiction between $\coh$ and~$\SC{agent}$.

The behaviour becomes allowed if we relax the consistency of SC orders and HSA
happens-before with $\coh$.  To understand how, note that the events $b$
and~$g$ are in the same work-group scope instance, and have the appropriate
\texttt{wg} scope tag.  Hence $b$ and~$g$ must be ordered by $\SC{wg}$, and, as
illustrated by Figure~\ref{t2wg}, the order must be $b \SC{wg} g$, which leads
to the sought contradiction with~\coh{}.

\begin{figure}[htp]
\caption{\label{t2wg}Relaxing further the consistency of SC orders and~\coh{}}
\vspace*{-5mm}
\begin{center}\moveback\fmt{t2+wg}
\end{center}
\vspace*{-5mm}
\end{figure}
Note that $b \SC{wg} g$ stems from $b \hhb g$ by consistency of $\SC{wg}$
and~\hhb.

\subsection{Tony3}
The following test has the same scope tree as the previous~\ltst{Tony2},
\emph{i.e.} there are two work-groups, $\{\myth{0}, \myth{2}\}$
and~$\{\myth{1}\}$; and one agent $\{\myth{0}, \myth{1}, \myth{2}\}$.  It has
no significant~\hhb{} relation as there are no \rf{}~in the execution of
interest.

{\footnotesize
\verbatiminput{img/t3.litmus}
}

This test is forbidden by our models. This is rather comforting as the test is
not sequentially consistent ($\po \cup \coh$ has a cycle), and that using
synchronising stores and loads should restore sequential consistency.
%{\color{blue} (is that the intent?).}

However, restoring SC with synchronising accesses becomes a bit subtle in the
presence of several scope instances.

First, we examine our test under our first option for defining the domain of SC
orders, \emph{i.e.} we consider that a given SC order is defined over all the
synchronising events emitted by the units in a given scope instance (option
{\tt tag2scope} in the definition of {\tt sync-instance}, see
Section~\ref{sc-orders}). As for~\ltst{Tony2} our model rejects the test by the consistency of~\SC{agent}
(defined over all events here) and~\coh{}.  Figure~\ref{t3fst} shows such a
contradiction.

\begin{figure}[h!]
\caption{\label{t3fst}Relaxing the consistency of SC orders
and~\coh{} for the test~\ltst{Tony3}}
\vspace*{-5mm}
\begin{center}\moveback\fmt{t3+fst}
\end{center}
\vspace*{-5mm}
\end{figure}

Then, we consider our second option for SC orders, \emph{i.e.} we consider that
a given SC order is defined over the events that specify a given scope instance
(option {\tt same-instance} in the definition of {\tt sync-instance}, see
Section~\ref{sc-orders}).

\begin{figure}[h!]
\caption{\label{t3}Relaxing the pairwise consistency of SC order for test~\ltst{Tony3}}
\vspace*{-5mm}
\begin{center}\moveback\fmt{t3}
\end{center}
\vspace*{-5mm}
\end{figure}
Figure~\ref{t3} illustrates that the pairwise consistency of SC orders is
instrumental in forbidding the test.  Namely, by the consistency of SC order
with \po{} and~\coh, we have $b \SC{agent} e$ (all intermediate events are in
the same agent scope instance and bear appropriate scope annotations);
similarly we have $e \SC{wg} b$, and thus reach an inconsistency.



