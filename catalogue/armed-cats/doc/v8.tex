\makeatletter
\@ifundefined{bwtrue}{\newif\ifbw\bwfalse}{}
\@ifundefined{acmtrue}{\newif\ifacm\acmtrue}{}
\makeatother
\makeatletter
\let\@period=\,
\makeatother
\documentclass[a4paper]{article}
\usepackage{xspace}
\usepackage{color}
\usepackage[leqno]{amsmath}
\usepackage{amssymb}
\usepackage{array}
\usepackage{tabularx}
\usepackage{booktabs}
\usepackage{url}
\usepackage{hyperref}
\usepackage{breakurl}
\usepackage{graphicx}
\usepackage{subfigure}
\usepackage{epsfig}
\usepackage[draft]{fixme}
\newif\ifcomments
\commentsfalse
\usepackage{url}
\makeatletter
\def\url@leostyle{%
  \@ifundefined{selectfont}{\def\UrlFont{\small\sf}}{\def\UrlFont{\small\sf}}}
\makeatother
\urlstyle{leo}
\usepackage[normalem]{ulem}
\usepackage{xspace}
%same notation policy for each instance of figure, thm, etc
\newcommand{\myfig}{Fig.~}
\newcommand{\mytab}{Tab.~}
\newcommand{\mysec}{Sec.~}
\newcommand{\mychap}{Chap.~}
\newcommand{\myapp}{App.~}
\newcommand{\etal}{et~al.\xspace}
\newcommand{\wrt}{w.r.t.\xspace}
\newcommand{\eg}{e.g.\xspace}
\newcommand{\ie}{i.e.\xspace}
\newcommand{\etc}{etc.\xspace}
\newcommand{\cf}{see\xspace}

%les maths comme au lycee
\let\eset\emptyset
\newcommand{\rplus}[1]{\mathop{{#1}^{+}}}
\newcommand{\rstar}[1]{\mathop{{#1}^{*}}}
\newcommand{\transc}[1]{\mathop{{#1}^{+}}}
\newcommand{\maybe}[1]{\mathop{{#1}^{?}}}

%def
  \newcommand{\dfnshort}[2]{{#1}\triangleq{#2}}
  \newcommand{\dfn}[2]{{#1}~\triangleq~{#2}}

%sets
\newcommand{\setcal}[1]{\mathcal{#1} }
\newcommand{\setbf}[1]{\mathbb{#1} }
\newcommand{\setfrak}[1]{\mathfrak{#1} }


  %writes, reads, barriers
  \let\setset\setbf
  \newcommand{\evts}{\setset{E}}

  \newcommand{\lwf}{\textsf{fence}}
  \newcommand{\mfence}{\ensuremath{\mathsf{mfence}}}
  \newcommand{\lwfence}{\ensuremath{\mathsf{fence}}}
  \newcommand{\ff}{\textsf{fence.sc}}
  \newcommand{\ffence}{\ensuremath{\mathsf{f{\kern-0.1em}fence}}}
  \newcommand{\fences}{\ensuremath{\mathsf{fence}}}
  \newcommand{\relw}{\ensuremath{\mathsf{rel}}}
  \newcommand{\acqr}{\ensuremath{\mathsf{acq}}}
  \newcommand{\cumul}{\ensuremath{\mathsf{cumul}}}
  \newcommand{\Acumul}{\ensuremath{\mathsf{A{\kern-0.3em}-{\kern-0.3em}cumul}}}

  %procs
  \newcommand{\pr}{\operatorname{proc}}

  %locs
  \newcommand{\lo}{\ell}
  \newcommand{\loc}{\operatorname{addr}}

  %vals 
  \newcommand{\val}{\operatorname{val}}

%relations
\newcommand{\stacklabel}[1]
%{\stackrel{\smash{\scriptscriptstyle\#1}}}
{\stackrel{\smash{\scriptstyle\textnormal{#1}}}}
  %any
  \newcommand{\rln}[1]{\ensuremath{\xrightarrow{#1}}}

  %po
  \newcommand{\iico}{\ensuremath{\mathsf{ii}}}
  \newcommand{\ctrlcfence}{\ensuremath{\mathsf{ctrlcfence}}}
  \newcommand{\ctrl}{\ensuremath{\mathsf{ctrl}}}
  \newcommand{\data}{\ensuremath{\mathsf{data}}}
  \newcommand{\addr}{\ensuremath{\mathsf{addr}}}
  \newcommand{\ddreg}{\ensuremath{\mathsf{dd}}}
  \newcommand{\rfreg}{{\mathsf{rfreg}}}
  \newcommand{\isb}{\ensuremath{\mathsf{isb}}}
  \newcommand{\isync}{\ensuremath{\mathsf{isync}}}
  \newcommand{\po}{\ensuremath{\mathsf{po}}}
  \newcommand{\ppo}{\ensuremath{\mathsf{ppo}}}
  %rf
  \newcommand{\rf}{\ensuremath{\mathsf{rf}}}
  %rfe
  \newcommand{\rfe}{\ensuremath{\mathsf{rfe}}}
  %rfi
  \newcommand{\rfi}{\ensuremath{\mathsf{rfi}}}
  %co
  \newcommand{\co}{\ensuremath{\mathsf{co}}}
  \newcommand{\coi}{\ensuremath{\mathsf{coi}}}
  \newcommand{\ews}{\ensuremath{\mathsf{coe}}}
  %fr
  \newcommand{\fr}{\ensuremath{\mathsf{fr}}}
  \newcommand{\ifr}{\ensuremath{\mathsf{fri}}}
  \newcommand{\efr}{\ensuremath{\mathsf{fre}}}
  %hb
  \newcommand{\com}{\ensuremath{\mathsf{com}}}
  \newcommand{\cause}{\ensuremath{\mathsf{cause}}}
  \newcommand{\hb}{\ensuremath{\mathsf{hb}}}
\newcommand{\propbase}{\operatorname{\ensuremath{\mathsf{prop{\kern-0.00em}-{\kern-0.1em}base}}}}
\newcommand{\prop}{\operatorname{\ensuremath{\mathsf{prop}}}}
  %poloc
  \newcommand{\polocllh}{\operatorname{\ensuremath{\mathsf{po{\kern-0.00em}-{\kern-0.1em}loc{\kern-0.00em}-{\kern-0.05em}llh}}}}
  \newcommand{\poloc}{\operatorname{\ensuremath{\mathsf{po{\kern-0.00em}-{\kern-0.05em}loc}}}}
  %criteria
  \newcommand{\acyclic}{\operatorname{acyclic}}
  \newcommand{\reflexive}{\operatorname{reflexive}}
  \newcommand{\irrefl}{\operatorname{irreflexive}}
  %po restricted to some events
  \newcommand{\WW}{\operatorname{\ensuremath{\mathsf{{WW}}}}}
  \newcommand{\RM}{\operatorname{\ensuremath{\mathsf{RM}}}}
  \newcommand{\RW}{\operatorname{\ensuremath{\mathsf{RW}}}}
  \newcommand{\RR}{\operatorname{\ensuremath{\mathsf{RR}}}}
  \newcommand{\RB}{\operatorname{\ensuremath{\mathsf{RB}}}}
  \newcommand{\MW}{\operatorname{MW}}
  \newcommand{\WM}{\operatorname{WM}}
  \newcommand{\WR}{\operatorname{\ensuremath{\mathsf{WR}}}}

\newcommand{\wide}{wide}
\newcommand{\wider}{wider}
\newcommand{\widest}{widest}
\newcommand{\narrow}{narrow}
\renewcommand{\narrower}{narrower}
\newcommand{\narrowest}{narrowest}
\newcommand{\narrowerop}{\operatorname{narrower}}

%%%Litmus
%%basic format
\let\prog\textsf
\let\as\texttt
\let\ltest\textbf

%% Conventional processor
\newcommand{\proc}[1]{\ensuremath{P_{#1}}}
%% Format graphs in columns
\makeatletter
\@ifundefined{useepstrue}{\newif\ifuseeps\useepsfalse}{}
\makeatother
\newlength{\fmtlength}
\ifuseeps
\newcommand{\fmtgraphcol}[3]
{\setlength{\fmtlength}{\linewidth}%
\addtolength{\fmtlength}{-#2}%
\framebox{\resizebox{\fmtlength}{#3}{\includegraphics{#1.eps}}}}
\else
\newcommand{\fmtgraphcol}[3]
{\setlength{\fmtlength}{\linewidth}%
\addtolength{\fmtlength}{-#2}%
\framebox{\resizebox{\fmtlength}{#3}{\input{#1.pstex_t}}}}
\fi
%%Format graph explicit width
\ifuseeps
\newcommand{\fmtgraph}[3]
{\resizebox{#2}{#3}{\includegraphics{#1.eps}}}
\newcommand{\tfmtgraph}[1]{\framebox{\includegraphics{#1.eps}}}
\else
\newcommand{\fmtgraph}[3]
{\resizebox{#2}{#3}{\input{#1.pstex_t}}}
\newcommand{\tfmtgraph}[1]{\framebox{\input{#1.pstex_t}}}
\fi
%%%%Test for figures as in the tutorial
% fix up fig2dev fonts to make actions in \sf
\iftrue
\makeatletter
% for the fig2dev version on P laptop
\gdef\SetFigFontNFSS#1#2#3#4#5{%
  \reset@font\fontsize{#1}{#2pt}%
%  \fontfamily{#3}\fontseries{#4}\fontshape{#5}%
  \fontfamily{\sfdefault}\fontseries{#4}\fontshape{#5}%
  \selectfont}%
% for the fig2dev version on P desktop
\gdef\SetFigFont#1#2#3#4#5{%
  \reset@font\fontsize{#1}{#2pt}%
%  \fontfamily{#3}\fontseries{#4}\fontshape{#5}%
  \fontfamily{\sfdefault}\fontseries{#4}\fontshape{#5}%
  \selectfont}
\makeatother
\newcommand{\asm}[1]{\texttt{#1}}
\newcommand{\myth}[1]{\ensuremath{\mathsf{T}_#1}}
\newcommand{\mylegend}[2]{#1}
\ifbw\newcommand{\bw}{-bw}\else\newcommand{\bw}{}\fi
\ifuseeps
\newcommand{\locfmt}[1]{\includegraphics{#1.eps}}
\newcommand{\newfmt}[1]{\includegraphics{img/#1.eps}}
\newcommand{\insfmt}[1]{\scalebox{0.28}{\includegraphics{ins/#1.eps}}}
\else
\newcommand{\locfmt}[1]{\input{#1.pstex_t}}
\newcommand{\newfmt}[1]{\input{img/#1\bw.pstex_t}}
\newcommand{\insfmt}[1]{\scalebox{0.28}{\input{ins/#1\bw.pstex_t}}}
\fi
\fi

\newcommand{\NEW}[1]{\textcolor{blue}{#1}}

\newcolumntype{Y}{@{}r@{\,}X}
%%Decorate instruction with a label
\newcommand{\instab}[2]{\ \(#2\)  & \as{#1}}
%%Pseudo-code load & store
\newcommand{\pset}[2]{\(\as{#1} \leftarrow \as{#2}\)}
\newcommand{\pstore}[2]{\pset{#2}{#1}}
\newcommand{\pload}[2]{\pset{#1}{#2}}
%%trick to add vertical space somewhere
\newcommand{\haut}{\rule{0ex}{2ex}}
\newcommand{\bas}{\rule[-1ex]{0.5ex}{0ex}}



\begin{document}
\title{Towards ARMv7 and ARMv8 models}
\author{Jade Alglave, Keryan Didier and Luc Maranget}
\maketitle
\let\prog\textsf

In this note we report on our progress towards modelling ARMv8.


Our model is written in an enhanced
version of the {\tt cat} language of Alglave and Maranget
(see~\cite{amt14}).
We describe the new \texttt{cat} languauge
at~\url{diy.inria.fr/tst7/doc/herd.html}).  \fixme{jade@Luc:
virer tst7 de l'url? Pour le moment on garde, c,a changera plus tard...
Doit-on changer diy.inria.fr en autre chose~?}
This language is a domain specific language for writing
consistency models such as the ones implemented by multiprocessors. Writing our
models in this language allows us to use them as input to the {\sf herd} tool
(see~\cite{amt14} and
\url{virginia.cs.ucl.ac.uk/herd} for the new versiob),
which thus becomes a simulator for the input
model.

We have tested our model against hardware chips (both v7 and v8), which we
detail below.  Our tests are automatically generated using our {\tt diy} tool
(see~\cite{alg10,ams10,ams12,amt14} and \url{diy.inria.fr/doc/gen.html}), and
run using the {\tt litmus} tool (see~\cite{ams11} and
\url{diy.inria.fr/doc/litmus.html}).

\paragraph{With respect to ARMv7} \fixme{jade@Luc: pourrais-tu mettre les refs
des chips v7 du papier des chats?  Non faire re'fe'rence a` ce papier et basta!}

\paragraph{With respect to ARMv8 $\!\!\!\!\!\!$} so far we have
experimented with:
\begin{description}
\item[\textbf{Denver64}]
A Google nexus9 tablet, whose processor is a dual core 64bit
``Denver'' NVIDIA Tegra K1.

\item[\textbf{A8X}]
An Apple iPad~air~2, whose processor is a 3-core
A8X Apple processor.

\item[\textbf{Snapdragon810}]
A LG~G~Flex~2 telephone, whose processor is a 8-core Qualcomm
Snapdragon810.
\end{description}

We distribute our model and our tests at \url{virginia.cs.ucl.ac.uk/herd} and
record our testing logs at \url{diy.inria.fr/aarch64/}.


\clearpage

\section{Verbatim of our tentative ARM model}

\subsection{Utilities and skeleton}

\begin{figure}[!h]
\begin{center}
\verbatiminput{armv8-utils.cat}
\end{center}
\caption{Utilities}
\end{figure}

\begin{figure}[!h]
\begin{center}
\verbatiminput{armv8-skeleton.cat}
\end{center}
\caption{Skeleton}
\end{figure}

\subsection{Preserved program order}

\begin{figure}[!h]
\verbatiminput{armv8-ppo.cat}
\caption{Preserved program order}
\end{figure}

\subsection{Fences}

\begin{figure}[!h]
\verbatiminput{armv8-fences.cat}
\caption{Fences}
\end{figure}

\clearpage

\subsection{Happens-before order}

\begin{figure}[!h]
\verbatiminput{armv8-happens-before.cat}
\caption{Happens-before}
\end{figure}

\subsection{Observation}

\begin{figure}[!h]
\verbatiminput{armv8-observation}
\caption{Observation}
\end{figure}

\subsection{Propagation order}

\begin{figure}[!h]
\verbatiminput{armv8-propagation.cat}
\caption{Propagation order}
\end{figure}

\clearpage

\section{SC per location \label{sec:sc-per-loc}}

\begin{figure}[!h]
\verbatiminput{armv8-sc-per-location.cat}
\vspace*{-4mm}
\caption{SC per location fragment of the ARM cat file}
\end{figure}

\textsc{sc per location} ensures that the communications $\com$ cannot
contradict $\poloc$ (program order between events relative to the same memory
location), \ie $\acyclic({\poloc} \cup {\com})$. This axiom forbids exactly (as
shown in~\cite[A.3 p.~184]{alg10}) the five scenarios \textsf{coWW, coRW1,
coRW2, coWR, coRR} given in \myfig\ref{fig:co}.

\begin{figure}[!h]
\begin{center}
\begin{tabular}{m{.3\linewidth} m{.3\linewidth} m{.3\linewidth}}
%\fmtgraph{coww}{.4\linewidth}{!}
\newfmt{coww}
&
%\fmtgraph{corw1}{.4\linewidth}{!}
\newfmt{corw1}
&
%\fmtgraph{corw2}{.7\linewidth}{!}
\newfmt{corw2}
\end{tabular}
\end{center}
\begin{center}
\begin{tabular}{m{.5\linewidth} m{.5\linewidth}}
%\fmtgraph{cowr}{.45\linewidth}{!}
\newfmt{cowr}
&
%\fmtgraph{corr}{.45\linewidth}{!}
\newfmt{corr}
\end{tabular}
\end{center}
\vspace*{-5mm}
\caption{The five scenarios forbidden by \textsc{sc per location}\label{fig:co}}
\end{figure}

The scenario \textsf{coWW} forces two writes to the same memory location $x$ in
program order to be in the same order in the coherence order $\co$. The
scenario \textsf{coRW1} forbids a read from $x$ to read from a $\po$-subsequent
write.  The scenario \textsf{coRW2} forbids the read $a$ to read from a write
$c$ which is $\co$-after a write $b$, if $b$ is $\po$-after $a$. The scenario
\textsf{coWR} forbids a read $b$ to read from a write $c$ which is $\co$-before
a previous write $a$ in program order. The scenario \textsf{coRR} imposes that
if a read $a$ reads from a write $c$, all subsequent reads in program order
from the same location (\eg the read $b$) read from $c$ or a $\co$-successor
write.

%On the web interface:
%\begin{itemize}
%\item \textsf{coWW}: \url{http://virginia.cs.ucl.ac.uk/herd-web-prerelease/?book=armed-cats&language=arm&cat=arm-from-paper&litmus=coWW}
%\item \textsf{coRW1}: \url{http://virginia.cs.ucl.ac.uk/herd-web-prerelease/?book=armed-cats&language=arm-from-paper&cat=arm&litmus=coRW1}
%\item \textsf{coRW2}: \url{http://virginia.cs.ucl.ac.uk/herd-web-prerelease/?book=armed-cats&language=arm&cat=arm-from-paper&litmus=coRW2}
%\item \textsf{coWR}: \url{http://virginia.cs.ucl.ac.uk/herd-web-prerelease/?book=armed-cats&language=arm&cat=arm-from-paper&litmus=coWR}
%\item \textsf{coRR}: \url{http://virginia.cs.ucl.ac.uk/herd-web-prerelease/?book=armed-cats&language=arm&cat=arm-from-paper&litmus=coRR}
%\end{itemize}

\clearpage

\section{No thin air \label{sec:no-thin-air}}

\begin{figure}[!h]
\verbatiminput{armv8-happens-before.cat}
\vspace*{-7mm}
\caption{No thin air fragment of the ARM cat file}
\end{figure}

\textsc{no thin air} defines a \emph{happens-before} relation $\hb$, \ie an
event $e_1$ happens before another event $e_2$ if they are in preserved program
order (which implies that $e_1$ is a read), or $e_1$ is a read and there is a
fence in program order between them, or $e_2$ reads from $e_1$. \textsc{no thin
air} requires the happens-before relation to be acyclic, which prevents values
from appearing out of thin air. This axiom typically forbids load buffering
scenarios, some of which we detail below.

\subsection{\`A la v7, i.e. {\sf lb+ppos} or {\sf lb+fences}} 

Consider the idiom \textsf{lb+ppos} in \myfig\ref{fig:lb}.

\begin{figure}[!h]
\vspace*{-2mm}
\begin{center}
%\fmtgraph{lb}{.24\linewidth}{!}
\newfmt{lb+ppos}
\end{center}
\vspace*{-8mm}
\caption{The load buffering scenario \textsf{lb} with ppo on both sides (forbidden) \label{fig:lb}}
\vspace*{-2mm}
\end{figure}

\myth{0} reads from $x$ and writes to $y$, imposing (for example) an address
dependency between the two accesses, so that they cannot be reordered.
Similarly \myth{1} reads from $y$ and (dependently) writes to $x$.  \textsc{no
thin air} ensures that the two threads cannot communicate in a manner that
creates a happens-before cycle, with the read from $x$ on \myth{0} reading from
the write to $x$ on \myth{1}, whilst \myth{1} reads from \myth{0}'s write to
$y$. Placing fences on each thread (e.g. {\tt dmb.ld}) would forbid the
behaviour just as well.

%On the web interface:
%\begin{itemize}
%\item \textsf{lb+addrs}: \url{http://virginia.cs.ucl.ac.uk/herd-web-prerelease/?book=armed-cats&language=arm&cat=arm-from-paper&litmus=LB+addrs}
%\item \textsf{lb+addrs}: \url{http://virginia.cs.ucl.ac.uk/herd-web-prerelease/?book=armed-cats&language=arm&cat=arm-from-paper&litmus=LB+dmbs}
%\item \textsf{lb+addrs}: \url{http://virginia.cs.ucl.ac.uk/herd-web-prerelease/?book=armed-cats&language=arm&cat=arm-from-paper&litmus=LB+dsb+addr}
%\end{itemize}

Note that address and data dependencies are observationally different.

\begin{figure}[h]
\vspace*{-2mm}
\begin{center}
\newfmt{lb+addrs+ww}
\end{center}
\vspace*{-8mm}
\caption{The \textsf{lb+addrs+ww} variant of the load buffering pattern \textsf{lb} (forbidden) \label{fig:lb+addrs+ww}}
\vspace*{-5mm}
\end{figure}

Consider the variant \textsf{lb+addrs+ww} in \myfig\ref{fig:lb+addrs+ww}: it is
forbidden by our definition of {\tt ppo}, since we put {\tt addr;po} in {\tt
ppo}. However, if we replace address dependencies by data dependencies in
\myfig\ref{fig:lb+addrs+ww}, to produce the \textsf{lb+datas+ww} variant, the
scenario is allowed by our model, and observed on ARM hardware (see
\url{http://diy.inria.fr/cats/data-addr/}).

%On the web interface:
%\begin{itemize}
%\item \textsf{lb+addrs+ww}: \url{http://virginia.cs.ucl.ac.uk/herd-web-prerelease/?book=armed-cats&language=arm&cat=arm-from-paper&litmus=LB+addrs+WW}
%\item \textsf{lb+addrs}: \url{http://virginia.cs.ucl.ac.uk/herd-web-prerelease/?book=armed-cats&language=arm&cat=arm-from-paper&litmus=LB+datas+WW}
%\end{itemize}

\subsection{\`A la v8, i.e. {\sf lb+acqs} or {\sf lb+rels}} 

Using v8 mechanisms can also forbid {\sf lb} behaviours. For example in
\myfig\ref{fig:lb-acqs} we've chosen to place acquire reads on each thread. 

\begin{figure}[!h]
\vspace*{-2mm}
\begin{center}
\newfmt{lb+acqs}
\end{center}
\vspace*{-8mm}
\caption{The load buffering scenario \textsf{lb} with acquires on both sides (forbidden) \label{fig:lb-acqs}}
\vspace*{-2mm}
\end{figure}

As shown in \myfig\ref{fig:lb-rels}, using release writes on each thread should
forbid the behaviour as well:

\begin{figure}[!h]
\vspace*{-2mm}
\begin{center}
\newfmt{lb+rels}
\end{center}
\vspace*{-8mm}
\caption{The load buffering scenario \textsf{lb} with releases on both sides (forbidden) \label{fig:lb-rels}}
\vspace*{-2mm}
\end{figure}

Of course one can also mix and match, placing for example an address dependency
on the first thread, and a release write on the second one. Any combination of
the choices we've shown above should forbid the {\sf lb} behaviour.

\newpage

\section{Observation \label{sec:observation}}

\begin{figure}[!h]
\verbatiminput{armv8-observation.cat}
\vspace*{-4mm}
\caption{Observation fragment of the ARM cat file}
\end{figure}

\textsc{observation} constrains the value of reads.  If a write $a$ to $x$ and
a write $b$ to $y$ are ordered by the propagation order $\prop$, and if a read
$c$ reads from the write of $y$, then any read $d$ from $x$ which happens after
$c$ (\ie $(c,d) \in \hb$) cannot read from a write that is $\co$-before the
write $a$ (\ie $(d,a) \not\in \fr$).

This axiom typically forbids the scenarios \textsf{mp+fence+ppo,
wrc+fence+ppo, isa2+fence+ppos} as shown in \myfig\ref{fig:mp},
\ref{fig:wrc} and \ref{fig:isa2}.

\subsection{Message passing}

\subsubsection{\`A la v7, i.e. \textsf{mp+fence+ppo}}

\begin{figure}[!h]
\vspace*{-2mm}
\begin{center}
\newfmt{mp+lwfence+ppo}
\end{center}
\vspace*{-8mm}
\caption{The message passing scenario \textsf{mp} with fence and
ppo (forbidden) \label{fig:mp}}
\end{figure}

\myth{0} writes a message in $x$, then sets up a flag in $y$, so that when
\myth{1} sees the flag (via its read from $y$), it can read the message in $x$.
For this scenario to behave as intended, following the message passing protocol
described above, the write to $x$ needs to be propagated to the reading thread
before the write to $y$.

The protocol would also be ensured with two fences (\textsf{mp+fences})

%On the web interface:
%\begin{itemize}
%\item \textsf{mp+dsb+addr}: \url{http://virginia.cs.ucl.ac.uk/herd-web-prerelease/?book=armed-cats&language=arm&cat=arm-from-paper&litmus=MP+dmb+addr}
%\item \textsf{mp+dmbs}: \url{http://virginia.cs.ucl.ac.uk/herd-web-prerelease/?book=armed-cats&language=arm&cat=arm-from-paper&litmus=MP+dmbs}
%\end{itemize}

\subsubsection{\`A la v8, i.e. \textsf{mp+rel+ppo}}

\begin{figure}[!h]
\vspace*{-2mm}
\begin{center}
\newfmt{mp+rel+ppo}
\end{center}
\vspace*{-8mm}
\caption{The message passing scenario \textsf{mp} with release and
ppo (forbidden) \label{fig:mp+rel+ppo}}
\end{figure}

\myth{0} writes a message in $x$, then sets up a flag in $y$, so that when
\myth{1} sees the flag (via its read from $y$), it can read the message in $x$.
For this scenario to behave as intended, following the message passing protocol
described above, the write to $x$ needs to be propagated to the reading thread
before the write to $y$.

%On the web interface:
%\begin{itemize}
%\item \textsf{mp+dsb+addr}: \url{http://virginia.cs.ucl.ac.uk/herd-web-prerelease/?book=armed-cats&language=arm&cat=arm-from-paper&litmus=MP+dmb+addr}
%\item \textsf{mp+dmbs}: \url{http://virginia.cs.ucl.ac.uk/herd-web-prerelease/?book=armed-cats&language=arm&cat=arm-from-paper&litmus=MP+dmbs}
%\end{itemize}

\newpage

\subsection{Write-to-read causality}

This scenario distributes the message passing over three threads instead of
two.  

\subsubsection{\`A la v7, i.e. \textsf{wrc+fence+ppo}}

In \myfig\ref{fig:wrc}, we chose to place a fence on \myth{1}, and illustrates
the A-cumulativity of the fence on \myth{1}, namely the {\tt cumulativity}
fragment of the cat file.

\begin{figure}[!h]
\begin{center}
%\fmtgraph{wrc}{.42\linewidth}{!}
\newfmt{wrc+lwfence+ppo}
\end{center}
\vspace*{-8mm}
\caption{The write-to-read causality scenario with fence and ppo (forbidden) \label{fig:wrc}}
\end{figure}

%On the web interface:
%\begin{itemize}
%\item \textsf{wrc+dsb+addr}: \url{http://virginia.cs.ucl.ac.uk/herd-web-prerelease/?book=armed-cats&language=arm&cat=arm-from-paper&litmus=WRC+dmb+addr}
%\item \textsf{wrc+dmb+addr}: \url{http://virginia.cs.ucl.ac.uk/herd-web-prerelease/?book=armed-cats&language=arm&cat=arm-from-paper&litmus=WRC+dsb+addr}
%\end{itemize}

%\newpage

\subsubsection{\`A la v8, i.e. \textsf{wrc+rel+ppo}}

In a more v8 manner, we chose in \myfig\ref{fig:wrc+rel+ppo} to place a release
write on \myth{1}, which forbids the behaviour just as well.

\begin{figure}[!h]
\begin{center}
%\fmtgraph{wrc}{.42\linewidth}{!}
\newfmt{wrc+rel+ppo}
\end{center}
\vspace*{-8mm}
\caption{The write-to-read causality scenario with release and ppo (forbidden) \label{fig:wrc+rel+ppo}}
\end{figure}

%On the web interface:
%\begin{itemize}
%\item \textsf{wrc+dsb+addr}: \url{http://virginia.cs.ucl.ac.uk/herd-web-prerelease/?book=armed-cats&language=arm&cat=arm-from-paper&litmus=WRC+dmb+addr}
%\item \textsf{wrc+dmb+addr}: \url{http://virginia.cs.ucl.ac.uk/herd-web-prerelease/?book=armed-cats&language=arm&cat=arm-from-paper&litmus=WRC+dsb+addr}
%\end{itemize}

\clearpage

\subsection{Power ISA2}

This scenario distributes the message passing over three threads like
\textsf{wrc}, but keeping the writes to $x$ and $y$ on the same thread.

\subsubsection{\`A la v7, i.e. \textsf{isa2+fence+ppos}}

Placing a fence on the first thread, and using dependencies on the last two,
should forbid the behaviour, as shown in \myfig\ref{fig:isa2}:

\begin{figure}[!h]
\begin{center}
%\fmtgraph{isa2}{.4\linewidth}{!}
\newfmt{isa2+lwfence+ppos}
\end{center}
\vspace*{-8mm}
\caption{The scenario \textsf{isa2} with fence and ppo twice (forbidden)\label{fig:isa2}}
\end{figure}

%On the web interface:
%\begin{itemize}
%\item \textsf{isa2+dmb+addr+addr}: \url{http://virginia.cs.ucl.ac.uk/herd-web-prerelease/?book=armed-cats&language=arm&cat=arm-from-paper&litmus=ISA2+dmb+addr+addr}
%\item \textsf{isa2+dsb+data+addr}: \url{http://virginia.cs.ucl.ac.uk/herd-web-prerelease/?book=armed-cats&language=arm&cat=arm-from-paper&litmus=ISA2+dsb+data+addr}
%\end{itemize}

\subsubsection{\`A la v8, i.e. \textsf{isa2+rel+ppos}}

On the first thread, replacing the fence with a release write should forbid the
behaviour just as well:

\begin{figure}[!h]
\begin{center}
%\fmtgraph{isa2}{.4\linewidth}{!}
\newfmt{isa2+rel+ppos}
\end{center}
\vspace*{-8mm}
\caption{The scenario \textsf{isa2} with release and ppo twice (forbidden)\label{fig:isa2+rel+ppos}}
\end{figure}

%On the web interface:
%\begin{itemize}
%\item \textsf{isa2+dmb+addr+addr}: \url{http://virginia.cs.ucl.ac.uk/herd-web-prerelease/?book=armed-cats&language=arm&cat=arm-from-paper&litmus=ISA2+dmb+addr+addr}
%\item \textsf{isa2+dsb+data+addr}: \url{http://virginia.cs.ucl.ac.uk/herd-web-prerelease/?book=armed-cats&language=arm&cat=arm-from-paper&litmus=ISA2+dsb+data+addr}
%\end{itemize}

\clearpage

\section{Propagation \label{sec:propagation}}

\begin{figure}[!h]
\verbatiminput{armv8-propagation.cat}
\vspace*{-8mm}
\caption{Propagation fragment of the ARM cat file}
\end{figure}

\textsc{propagation} constrains the order in which writes to memory are
propagated to the other threads, so that it does not contradict the coherence
order, \ie $\acyclic(\co \cup \prop)$. Fences sometimes constrain the
propagation of writes, as in the cases of \textsf{mp} (see \myfig\ref{fig:mp}),
\textsf{wrc} (see \myfig\ref{fig:wrc}) or \textsf{isa2} (see
\myfig\ref{fig:isa2}). They can also, in combination with the coherence order,
create new ordering constraints.

\subsection{Idiom \textsf{s}}

\subsubsection{\`A la v7, i.e. \textsf{s+fence+ppo}}

In the scenario~\textsf{s+fence+ppo}, \myth{1} reads from~$y$, reading the
value stored by the write~$b$, and then writes to~$x$. A fence
on~\myth{0} forces the write~$a$ to~$x$ to propagate to \myth{1} before the
write $b$ to~$y$.  Thus, as \myth{1} sees the write~$b$ by reading its value
(read~$c$) and as the write~$d$ is forced to occur by a dependency (\ppo) after
the read~$c$, that write~$d$ cannot \co-precede the write~$a$.

\begin{figure}[!h]
\begin{center}
\newfmt{s+lwfence+ppo}
\end{center}
\vspace*{-8mm}
\caption{The scenario \textsf{s} with fence and ppo (forbidden)\label{fig:s}}
\vspace*{-6mm}
\end{figure}

%On the web interface:
%\begin{itemize}
%\item \textsf{s+dmb+addr}: \url{http://virginia.cs.ucl.ac.uk/herd-web-prerelease/?book=armed-cats&language=arm&cat=arm-from-paper&litmus=S+dmb+addr}
%\item \textsf{s+dsb.st+addr}: \url{http://virginia.cs.ucl.ac.uk/herd-web-prerelease/?book=armed-cats&language=arm&cat=arm-from-paper&litmus=S+dsb.st+addr}
%\end{itemize}

\subsubsection{\`A la v8}

Replacing the fence by a release write should forbid the behaviour just as
well:

\begin{figure}[!h]
\begin{center}
\newfmt{s+rel+ppo}
\end{center}
\vspace*{-8mm}
\caption{The scenario \textsf{s} with release write and ppo (forbidden)\label{fig:s-rel}}
\end{figure}

\clearpage

\subsection{Idioms \textsf{2+2w} and \textsf{w+rw+2w}}

\subsubsection{\`A la v7, i.e. \textsf{2+2w+fences} and \textsf{w+rw+2w+fences}}

The \textsf{2+2w+fences} pattern (given in \myfig\ref{fig:2+2w}) is for us
the archetypal illustration of coherence order and fences interacting to yield
new ordering constraints. This scenario is forbidden.

\begin{figure}[!h] 
\vspace*{-8mm}
\begin{center}
\newfmt{2+2w+lwfences} 
\end{center}
\vspace*{-8mm}
\caption{The scenario \textsf{2+2w+fences}\label{fig:2+2w} (forbidden)} 
\end{figure}

The \textsf{w+rw+2w+fences} scenario in \myfig\ref{fig:w+rw+2w} distributes
\textsf{2+2w+fences} over three threads.  This scenario is to
\textsf{2+2w+fences} what \textsf{wrc} is to \textsf{mp}. Thus, just as in the
case of \textsf{mp} and \textsf{wrc}, the fence plays a cumulative role, which
ensures that \textsf{2+2w} and \textsf{w+rw+2w} respond to the fence in the
same way.

\begin{figure}[!h] 
\begin{center}
\newfmt{wrw+2w+lwsyncs}
\end{center}
\vspace*{-10mm}
\caption{The scenario \textsf{w+rw+2w+fences} \label{fig:w+rw+2w} (forbidden)} 
\end{figure}

%On the web interface:
%\begin{itemize}
%\item \textsf{2+2w+dmb+dmb.st}: \url{http://virginia.cs.ucl.ac.uk/herd-web-prerelease/?book=armed-cats&language=arm&cat=arm-from-paper&litmus=2+2W+dmb+dmb.st}
%\item \textsf{2+2w+dmb+dsb}: \url{http://virginia.cs.ucl.ac.uk/herd-web-prerelease/?book=armed-cats&language=arm&cat=arm-from-paper&litmus=2+2W+dmb+dsb}
%\end{itemize}

\subsubsection{\`A la v8}

Using release writes in place of fences in the \textsf{2+2w} pattern forbids
the behaviour, as shown below:

\begin{figure}[!h] 
\vspace*{-8mm}
\begin{center}
\newfmt{2+2w+rels} 
\end{center}
\vspace*{-8mm}
\caption{The scenario \textsf{2+2w+rels}\label{fig:2+2w+rels} (forbidden)} 
\end{figure}

\clearpage

For {\sf w+rw+2w}, one can for example protect the communication between
\myth{0} and \myth{1} with release and acquire accesses:

\begin{figure}[!h] 
\begin{center}
\newfmt{wrw+2w+rel+acq-rel+fence}
\end{center}
\vspace*{-8mm}
\caption{The scenario \textsf{w+rw+2w+rel+acq-rel+fence} \label{fig:w+rw+2w}
(forbidden)} 
\end{figure}

\clearpage

\subsection{Store buffering}

\subsubsection{\`A la v7, i.e. (\textsf{sb+fences})}

Consider the store buffering (\textsf{sb+fences}) scenario given in
\myfig\ref{fig:sb}.

\begin{figure}[!h]
\begin{center}
%\fmtgraph{sb}{.23\linewidth}{!}
\newfmt{sb+ffences}
\end{center}
\vspace*{-5mm}
\caption{The store buffering scenario \textsf{sb} with fences (forbidden)
\label{fig:sb}}
\end{figure}

We need a fence on each side to prevent the reads $b$ and $d$ from reading the
initial state.

%On the web interface:
%\begin{itemize}
%\item \textsf{sb+dmb+dmb.st}: \url{http://virginia.cs.ucl.ac.uk/herd-web-prerelease/?book=armed-cats&language=arm&cat=arm-from-paper&litmus=SB+dmb+dmb.st}
%\item \textsf{sb+dmb+dsb}: \url{http://virginia.cs.ucl.ac.uk/herd-web-prerelease/?book=armed-cats&language=arm&cat=arm-from-paper&litmus=SB+dmb+dsb}
%\end{itemize}

\subsubsection{\`A la v8, i.e. \textsf{sb+rel-acq}}

Consider the store buffering (\textsf{sb+ra}) scenario given in
\myfig\ref{fig:sb-ra}.

\begin{figure}[!h]
\begin{center}
%\fmtgraph{sb}{.23\linewidth}{!}
\newfmt{sb+ra}
\end{center}
\vspace*{-5mm}
\caption{The store buffering scenario \textsf{sb} with release-acquire (forbidden)
\label{fig:sb-ra}}
\end{figure}

Making all accesses release (for writes) and acquire (for reads) prevent the
reads $b$ and $d$ from reading the initial state.

%On the web interface:
%\begin{itemize}
%\item \textsf{sb+dmb+dmb.st}: \url{http://virginia.cs.ucl.ac.uk/herd-web-prerelease/?book=armed-cats&language=arm&cat=arm-from-paper&litmus=SB+dmb+dmb.st}
%\item \textsf{sb+dmb+dsb}: \url{http://virginia.cs.ucl.ac.uk/herd-web-prerelease/?book=armed-cats&language=arm&cat=arm-from-paper&litmus=SB+dmb+dsb}
%\end{itemize}

\clearpage

\subsection{Read-to-write causality}

The read-to-write causality scenario \textsf{rwc+fences} (\cf
\myfig\ref{fig:rwc}) distributes the \textsf{sb} scenario over three threads
with a read $b$ from $x$ between the write $a$ of $x$ and the read $c$ of $y$.

\subsubsection{\`A la v7, i.e. \textsf{rwc+fences}}

It is to \textsf{sb} what \textsf{wrc} is to \textsf{mp}, thus responds to
fences in the same way as \textsf{sb}. Hence it needs two fences too.

\begin{figure}[!h]
\begin{center}
%\fmtgraph{rwc}{.4\linewidth}{!}
\newfmt{rwc+syncs}
\end{center}
\vspace*{-5mm}
\caption{The read-to-write causality scenario \textsf{rwc} with fences (forbidden) \label{fig:rwc}}
\end{figure}

%\begin{itemize}
%\item \textsf{rwc+dmb+dmb.st}: \url{http://virginia.cs.ucl.ac.uk/herd-web-prerelease/?book=armed-cats&language=arm&cat=arm-from-paper&litmus=RWC+dmb+dmb.st}
%\item \textsf{rwc+dmb+dsb}: \url{http://virginia.cs.ucl.ac.uk/herd-web-prerelease/?book=armed-cats&language=arm&cat=arm-from-paper&litmus=RWC+dmb+dsb}
%\end{itemize}

\subsubsection{\`A la v8, i.e. \textsf{rwc+ra}}

The idiom {\sf rwc} responds to release-acquire in the same way as \textsf{sb}. 

\begin{figure}[!h]
\begin{center}
%\fmtgraph{rwc}{.4\linewidth}{!}
\newfmt{rwc+ra}
\end{center}
\vspace*{-5mm}
\caption{The read-to-write causality scenario \textsf{rwc} with release-acquire (forbidden) \label{fig:rwc-ra}}
\end{figure}

%\begin{itemize}
%\item \textsf{rwc+dmb+dmb.st}: \url{http://virginia.cs.ucl.ac.uk/herd-web-prerelease/?book=armed-cats&language=arm&cat=arm-from-paper&litmus=RWC+dmb+dmb.st}
%\item \textsf{rwc+dmb+dsb}: \url{http://virginia.cs.ucl.ac.uk/herd-web-prerelease/?book=armed-cats&language=arm&cat=arm-from-paper&litmus=RWC+dmb+dsb}
%\end{itemize}

\clearpage

\subsection{Independent reads of independent writes}

\subsubsection{\`A la v7, i.e. \textsf{iriw+fences}}

ARM documentation~\cite{arm:cookbook} forbids \textsf{iriw+dmb}
(\cf\myfig\ref{fig:iriw}).

\begin{figure}[!h]
\begin{center}
\newfmt{iriw+ffences}
\end{center}
\caption{The independent reads from independent writes scenario \textsf{iriw}
with fences (forbidden)\label{fig:iriw}}
\end{figure}

%\begin{itemize}
%\item \textsf{iriw+dmbs}: \url{http://virginia.cs.ucl.ac.uk/herd-web-prerelease/?book=armed-cats&language=arm&cat=arm-from-paper&litmus=IRIW+dmbs}
%\end{itemize}

\subsubsection{\`A la v8, i.e. (\textsf{iriw+ra})}

Placing release-acquire everywhere restores SC, thus forbids \textsf{iriw}:

\begin{figure}[!h]
\begin{center}
\newfmt{iriw+ra}
\end{center}
\caption{The independent reads from independent writes scenario \textsf{iriw}
with release-acquire (forbidden)\label{fig:iriw-ra}}
\end{figure}

%\begin{itemize}
%\item \textsf{iriw+dmbs}: \url{http://virginia.cs.ucl.ac.uk/herd-web-prerelease/?book=armed-cats&language=arm&cat=arm-from-paper&litmus=IRIW+dmbs}
%\end{itemize}

\clearpage

\subsection{Idiom \textsf{r}}

In the \textsf{r} scenario, the thread \myth{0} writes to~$x$ (event~$a$) and
then to~$y$ (event~$b$). \myth{1} writes to $y$ and reads from $x$.

\subsubsection{\`A la v7, i.e. \textsf{r+fences}}

A fence on each thread forces the write $a$ to $x$ to propagate to \myth{1}
before the write $b$ to $y$. Thus if the write $b$ is $\co$-before the write
$c$ on \myth{1}, the read $d$ of $x$ on \myth{1} cannot read from a write that
is $\co$-before the write $a$.

\begin{figure}[!h]
\begin{center}
\newfmt{r+ffences}
\end{center}
\vspace*{-5mm}
\caption{The scenario \textsf{r} with fences (forbidden)\label{fig:r}}
\end{figure}

%\begin{itemize}
%\item \textsf{r+dmb+dmb.st}: \url{http://virginia.cs.ucl.ac.uk/herd-web-prerelease/?book=armed-cats&language=arm&cat=arm-from-paper&litmus=R+dmb+dmb.st}
%\item \textsf{r+dmb+dsb}: \url{http://virginia.cs.ucl.ac.uk/herd-web-prerelease/?book=armed-cats&language=arm&cat=arm-from-paper&litmus=R+dmb+dsb}
%\item \textsf{r+dmbs}: \url{http://virginia.cs.ucl.ac.uk/herd-web-prerelease/?book=armed-cats&language=arm&cat=arm-from-paper&litmus=R+dmbs}
%\end{itemize}

\subsubsection{\`A la v8, i.e. \textsf{r+ra}}

Release or acquire accesses everywhere forces the write $a$ to $x$ to propagate
to \myth{1} before the write $b$ to $y$, which is enough to forbid the
behaviour. 

\begin{figure}[!h]
\begin{center}
\newfmt{r+ra}
\end{center}
\vspace*{-5mm}
\caption{The scenario \textsf{r} with release-acquire (forbidden)\label{fig:r-ra}}
\end{figure}

%\begin{itemize}
%\item \textsf{r+dmb+dmb.st}: \url{http://virginia.cs.ucl.ac.uk/herd-web-prerelease/?book=armed-cats&language=arm&cat=arm-from-paper&litmus=R+dmb+dmb.st}
%\item \textsf{r+dmb+dsb}: \url{http://virginia.cs.ucl.ac.uk/herd-web-prerelease/?book=armed-cats&language=arm&cat=arm-from-paper&litmus=R+dmb+dsb}
%\item \textsf{r+dmbs}: \url{http://virginia.cs.ucl.ac.uk/herd-web-prerelease/?book=armed-cats&language=arm&cat=arm-from-paper&litmus=R+dmbs}
%\end{itemize}

\clearpage

\bibliographystyle{plain} \bibliography{armed-cats}

\end{document}

\section{Verbatim of our old tentative ARMv7 model, i.e. without ldrex and
strex, rephrased from and equivalent to~\cite{amt14}}

\begin{verbatim}
"ARMv7 without ldrex/strex"

let com = rf | co | fr

let dp = addr | data
let rdw = po-loc & (fre;rfe)
let detour = po-loc & (coe ; rfe)

let ii0 = dp | rfi | rdw
let ic0 = 0
let ci0 = ctrlisb | detour
let cc0 = dp | ctrl | (addr;po)

let rec ii = ii0 | ci | (ic;ci) | (ii;ii)
and ic = ic0 | ii | cc | (ic;cc) | (ii;ic)
and ci = ci0 | (ci;ii) | (cc;ci)
and cc = cc0 | ci | (ci;ic) | (cc;cc)

let ppo = ii & (R*R) | ic & (R*W)

let all-dmb = dmb | dmb.st
let all-dsb = dsb | dsb.st
let strong-fence = all-dmb | all-dsb
let weak-fence = 0
let fence = strong-fence | weak-fence
let A-cumul = rfe;fences

let hb = ppo | fence | rfe

let prop-base = 
let prop = (fence | A-cumul);hb* & (W*W)
           | com*; prop-base*; strong-fence; hb*

\end{verbatim}

\clearpage

\begin{verbatim}
procedure sc-per-location() =
  acyclic po-loc | com
end

procedure no-thin-air() =
  acyclic hb
end

procedure observation() =
  irreflexive fre;obs
end

procedure propagation() =
  acyclic co | prop
end

procedure armv7-without-ldrex-strex() =
  call sc-per-location()
  call no-thin-air()
  call observation()
  call propagation()
end
\end{verbatim}

\clearpage

\section{Verbatim of our new tentative ARMv7 model, i.e. with ldrex and
strex}

\begin{verbatim}
"ARMv7 with ldrex/strex"

let com = rf | co | fr

let dp = addr | data
let rdw = po-loc & (fre;rfe)
let detour = po-loc & (coe ; rfe)

let ii0 = dp | rfi | rdw
let ic0 = 0
let ci0 = ctrlisb | detour
let cc0 = dp | ctrl | (addr;po)

let rec ii = ii0 | ci | (ic;ci) | (ii;ii)
and ic = ic0 | ii | cc | (ic;cc) | (ii;ic)
and ci = ci0 | (ci;ii) | (cc;ci)
and cc = cc0 | ci | (ci;ic) | (cc;cc)

let ppo = ii & (R*R) | ic & (R*W)

let all-dmb = dmb | dmb.st
let all-dsb = dsb | dsb.st
let strong-fence = all-dmb | all-dsb
let weak-fence = 0
let fence = strong-fence | weak-fence
let A-cumul = rfe;fences

let hb = ppo | fence | rfe

let prop-base = (fence | A-cumul);hb* & (W*W)
let prop = prop-base & W*W
           | com*; prop-base*; strong-fence; hb*
let xx = (W*W) & (X*X) & po
\end{verbatim}

\clearpage

\begin{verbatim}
procedure sc-per-location() =
  acyclic po-loc | com
end

procedure no-thin-air() =
  acyclic hb
end

procedure observation() =
  irreflexive fre;prop;hb*
end

procedure propagation() =
  acyclic co | prop
  acyclic co | xx 
end

procedure atomic() =
  empty rmw & (fre;coe)
end

procedure armv7-with-ldrex-strex() =
  call sc-per-location()
  call no-thin-air()
  call observation()
  call propagation()
  call atomic()
end
\end{verbatim}

\clearpage

\section{Verbatim of a first tentative ARMv8 model}
\fixme{jade: a revoir; peut etre que ca pourrait etre doc64, emballe dans des
procedures?}

\begin{verbatim}
"ARMv8"

let com = rf | co | fr

let dp = addr | data
let rdw = po-loc & (fre;rfe)
let detour = po-loc & (coe ; rfe)

let ii0 = dp | rfi | rdw
let ic0 = 0
let ci0 = ctrlisb | detour
let cc0 = dp | ctrl | (addr;po)

let rec ii = ii0 | ci | (ic;ci) | (ii;ii)
and ic = ic0 | ii | cc | (ic;cc) | (ii;ic)
and ci = ci0 | (ci;ii) | (cc;ci)
and cc = cc0 | ci | (ci;ic) | (cc;cc)

let acq-po = (Acq * M) & po
let po-rel = (M * Rel) & po

let ppo = ii & (R*R) | ic & (R*W) | acq-po

let all-dmb = dmb | dmb.st
let all-dsb = dsb | dsb.st
let strong-fence = all-dmb | all-dsb
let weak-fence = 0
let fence = strong-fence | weak-fence
let A-cumul = rfe;fences

let hb = ppo | fence | rfe

let prop-base = (fence | A-cumul);hb* & (W*W)
let car = com & (Acq | Rel) * (Acq | Rel)
let poar = po & (Acq | Rel) * (Acq | Rel)
let prop = prop-base & W*W
           | com*; prop-base*; strong-fence; hb*
           | car | poar
let xx = (W*W) & (X*X) & po
\end{verbatim}

\clearpage

\begin{verbatim}
procedure sc-per-location() =
  acyclic po-loc | com
end

procedure no-thin-air() =
  acyclic hb
end

procedure observation() =
  irreflexive fre;prop;hb*
end

procedure propagation() =
  acyclic co | prop
  acyclic co | xx 
end

procedure atomic() =
  empty rmw & (fre;coe)
end

procedure armv8() =
  call sc-per-location()
  call no-thin-air()
  call observation()
  call propagation()
  call atomic()
end
\end{verbatim}

\clearpage

\section{Verbatim of another tentative ARMv8 model, with ldrex/strex and v8
features, and {\tt restoring-sc} check} \fixme{jade: a revoir; plus rajouter
doc64 si jamais le precedent n'est pas doc64}

\begin{verbatim}
"ARM mit ldrex/strex and v8"

let com = rf | co | fr

let dp = addr | data
let rdw = po-loc & (fre;rfe)
let detour = po-loc & (coe ; rfe)

let ii0 = dp | rfi | rdw
let ic0 = 0
let ci0 = ctrlisb | detour
let cc0 = dp | ctrl | (addr;po)

let rec ii = ii0 | ci | (ic;ci) | (ii;ii)
and ic = ic0 | ii | cc | (ic;cc) | (ii;ic)
and ci = ci0 | (ci;ii) | (cc;ci)
and cc = cc0 | ci | (ci;ic) | (cc;cc)

let acq-po = (Acq * M) & po
let po-rel = (M * Rel) & po

let ppo = ii & R*R | ic & R*W | acq-po

let all-dmb = dmb.sy|dmb.st|dmb.ld
let all-dsb = dsb.sy|dsb.st|dsb.ld
let strong-fence = all-dmb | all-dsb
let weak-fence = po-rel
let fence = strong-fence | weak-fence
let A-cumul = rfe;fence

let hb = ppo | (R * M) & fence | rfe

let prop = ((fence | A-cumul);hb*) & W*W

let xx = (W*W) & (X *X) & po 

let car = com & (Acq|Rel) * (Acq|Rel)
let poar = po & (Acq|Rel) * (Acq|Rel)
\end{verbatim}

\clearpage

\begin{verbatim}

procedure sc-per-location() =
  acyclic po-loc | com
end

procedure no-thin-air() =
  acyclic hb
end

procedure observation() =
  irreflexive fre;prop;hb*
end

procedure propagation() =
  acyclic co | prop
end

procedure atomic() =
  empty rmw & (fre;coe)
end

procedure restoring-sc() =
  acyclic co | xx 
  acyclic ((com*;strong-fence)|car|(rfe?;weak-fence));hb* | poar
end

procedure arm() =
  call sc-per-location()
  call atomic()
  call no-thin-air
  call observation()
  call propagation()
  call restoring-sc()
end

call arm()
\end{verbatim}

\section{Restoring SC}

\begin{figure}[!h]
\begin{verbatim}
let xx = (W*W) & (X *X) & po 

let car = com & (Acq|Rel) * (Acq|Rel)
let poar = po & (Acq|Rel) * (Acq|Rel)

procedure restoring-sc() =
  acyclic co | xx
  acyclic ((com*;strong-fence)|car|(rfe?;weak-fence));hb* | poar
end
\end{verbatim}
\vspace*{-4mm}
\caption{Restoring SC fragment of the ARM cat file}
\end{figure}

\subsection{With full fences only}

\begin{figure}[!h]
\begin{verbatim}
procedure restoring-sc() =
  [...] 
  acyclic (com*;strong-fence) [...]
end
\end{verbatim}
\vspace*{-4mm}
\caption{Restoring SC with full fences only}
\end{figure}

\subsection{With acquire-release only}

\begin{figure}[!h]
\begin{verbatim}
let car = com & (Acq|Rel) * (Acq|Rel)
let poar = po & (Acq|Rel) * (Acq|Rel)

procedure restoring-sc() =
  [...]
  acyclic ([...]|car|(rfe?;weak-fence));hb* | poar
end
\end{verbatim}
\vspace*{-4mm}
\caption{Restoring SC with acquire-release only}
\end{figure}


